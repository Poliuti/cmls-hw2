\section{Flanger}
A clear comprehension of the flanger from a perceptive point of view  is highlighted by the frequency response of the comb filter. 
At the heart of the algorithm, comb filter is the simplest example of recursive delay network.
In fact, flanger is a recursive comb filter with interpolating and time-varying delay line. Changes in delay time correspond to the characteristic way peaks in the frequency response move up and down in frequency\cite{puckette2006theory}.

Comb filter is obtained feeding a $g_{FB}$ fraction of the output back to the input as we can see looking at (figura) its typical difference equation:

\[
	y[n] = g_{FB} y[n - M(n)] + x[n] + (g_{FF} - g_{FB}) x[n - M[n]],
\]
where $g_{FF}$ is the \textit{depth} of the flanger and $M[n]$ is a function describing variable delay times. In the case of interest it is represented by a Low Frequency Oscillator of different and selectable waveforms. Thus, the frequency response of the flanger is equal to 
\[
       \frac{Y[z]}{X[z]} = \frac{1 + z^{-M[n]} (g_{FF} - g_{FB})} {1 - z^{-M[n]}  g_{FB} }
\]

and looking at the absolute value of the frequency response of the non-recirculating comb filter with $g_{FB} = 0$ 
\[
    |H(z)| = \sqrt{ 1 + 2g_{FF} cos(\omega M[n]) + g_{FF}^1}
\]
 we can appreciate the classical spectrum with peaks and notches lying respectively in 
\[
       \omega_p = \frac{2 \pi p}{M},
\]
where $p = 0, 1, 2, ..., M-1$ and 
\[
       \omega_n = \frac{(2n+1) \pi}{M},
\]
where $n = 0, 1, 2, ..., M-1$. Now, it is easy to see the dependence between frequencies enhanced and delay times of the flanger. 